\documentclass[12pt, a4paper]{article}
\renewcommand*\contentsname{Inhaltsverzeichnis}
\usepackage[ngerman]{babel}
\usepackage{mathptmx}
\usepackage{blindtext}
\usepackage{emptypage}
\usepackage{wrapfig}
\usepackage[pdftex]{graphicx}
\usepackage{geometry}
\usepackage{setspace}
\usepackage{hyperref}
\usepackage[version=4,arrows=pgf-filled,
textfontname=sffamily,
mathfontname=mathsf]{mhchem}
\usepackage[table]{xcolor}
\usepackage{multirow}
\usepackage[table]{xcolor}
\usepackage{array}
\usepackage{float}
\usepackage{mathcomp}
\usepackage{csquotes}
\usepackage[backend=biber,style=chem-acs,sorting=none]{biblatex}
\addbibresource{literatur.bib}  % Deine .bib-Datei einbinden

\DeclareCiteCommand{\cite}
  {\usebibmacro{prenote}}
  {\textsuperscript{\printfield{labelnumber}}}
  {\multicitedelim}
  {\usebibmacro{postnote}}


 \geometry{
 a4paper,
 total={170mm,257mm},
 left=25mm,
 top=25mm,
 }
\setstretch{1.213}


\newcommand{\datum}{\day.\month.\year}
\DeclareGraphicsExtensions{.pdf,.jpeg,.png,.jpg} 

\begin{document}


\begin{figure}
    \includegraphics[scale=0.14]{Universität_Bayreuth.svg.png}
\end{figure}


%Deckblatt

{\raggedright Universität Bayreuth\\  95447 Bayreuth}


\vspace{5cm}

\begin{center}
{\LARGE\bf{Anorganische Chemie III}} \\  
\vspace{1cm}
{\Large\bf{Phasendiagramme und Polymorphie}}\\
\vspace{0.5cm}
{\large Justus Friedrich\\}
{Studiengang: B.Sc. Chemie\\}
{4. Fachsemester}
\end{center}





\thispagestyle{empty}
\begin{center}
{\small Matrikelnummer: 1956010 \\
E–Mail:  bt725206@myubt.de}
\end{center}

\vspace{5cm}

\today


\newpage
%Inhaltsverzeichnis
\tableofcontents
\thispagestyle{empty}


%Teil 1
\newpage
\setcounter{page}{1}
\section{Einleitung}



\subsection{Motivation}
Stoffe können sich stark in ihren chemischen und physikalischen Eigenschaften unterscheiden, selbst wenn sie dieselbe Chemische Zusammensetzung besitzten. Dies liegt häufig an der Tatsache, dass sie in unterschiedlichen Kristallstrukturen auftreten können. Diese Stoffe werden Polymorphe gennant. Bereits kleine Änderungen in den Synthesebedingungen können die Ausbildung verschiedener Phasen beeinflussen. \\
\noindent
Das Ziel dieses Versuchs ist es, gezielt metastabile Phasen von (\ce{CaCO3}) und Benzamid zu synthetisieren. Darüber hinaus soll untersucht werden, wie sich unterschiedliche Mischungsverhältnisse von Nickel und Antimon auf die entstehenden Phasen und deren Eigenschaften auswirken.\cite{Skript}


\newpage
%Teil2
\section{Durchführung}
\subsection{\texorpdfstring{Synthese von Ni\textsubscript{1±x}Sb\textsubscript{1}}{Synthese von Ni1±xSb1}}
Es werden, um die Acht verschiedene Zusammensetzungen herzustellen, gemäß der Tabelle \ref{Verhältnisse} Nickel und Sb in eine Quarzampulle gegeben. 
Die Quarzampulle wird evakumiert und abgeschmolzen. Anschließend wird die Quarzampulle bei 1100 °C für 1 Tag aufgeschmolzen und danach für 3 Tage bei 800 °C 
getempert.


\begin{table}[!h]
  \caption{Zeigt die Atom Verhältnisse des Produkt, und die dafür nötigen Eduktmassen und deren Mol Anzahl. Die Berrechungen für die Mol-Anzahl sind in Gleichung (1) und (2) dargestellt.}
  \begin{center}
    \begin{tabular}{|>{\centering\arraybackslash}p{2.3cm}|>{\centering\arraybackslash}p{1.1cm}|>{\centering\arraybackslash}p{1.1cm}|>{\centering\arraybackslash}p{1.1cm}|>{\centering\arraybackslash}p{1.1cm}|>{\centering\arraybackslash}p{1.1cm}|>{\centering\arraybackslash}p{1.1cm}|>{\centering\arraybackslash}p{1.1cm}|>{\centering\arraybackslash}p{1.1cm}|}
      \hline
      \rowcolor{gray}
      \cellcolor{lime}At\% \ce{Sb} & 97\% & 75\% & 60\% & 52\% & 50\% & 46.3\% & 40\% & 37\% \\
      \hline
      \rowcolor{yellow}
       \cellcolor{lime}Masse \ce{Sb} [g]&0.787&0.689&0.605&0.554&0.540&0.513&0.464&0.443 \\
      \hline
       \cellcolor{lime}Mol \ce{Sb} [mmol]&6.466& 5.661& 4.972& 4.547& 4.433& 4.214& 3.813& 3.608 \\
      \hline
      \rowcolor{gray}
       \cellcolor{lime}Mol\% \ce{Ni} & 3\% & 25\% & 40\% & 48\% & 50\% & 53.7\% & 60\% & 63\% \\
      \hline
      \rowcolor{yellow}      
       \cellcolor{lime}Masse \ce{Ni} [g]&0.012&0.111&0.194&0.246&0.260&0.287&0.335&0.364 \\
      \hline
       \cellcolor{lime}Mol \ce{Ni} [mmol] &0.200&1.887&3.315&4.197&4.433&4.888&5.720&6.203 \\
      \hline
    \end{tabular}
  \end{center}

  \label{Verhältnisse}
\end{table}

\subsection{Gleichungen zur Berechnung}
\begin{equation}
  \frac{2 g}{M(\ce{Sb})+\frac{mol\%  (\ce{Ni})}{mol\%  (\ce{Sb})}\cdot M(\ce{Ni})}= n(\ce{Sb})
\end{equation}

\begin{equation}
  \frac{2 g}{M(\ce{Sb})+\frac{mol\%  (\ce{Ni})}{mol\%  (\ce{Sb})}\cdot M(\ce{Ni})}\cdot \frac{mol\%  (\ce{Ni})}{mol\%  (\ce{Sb})}= n(\ce{Ni})
\end{equation}



\newpage
\section{Auswertung}
\subsection{Ergebnisse}
\subsection{Diskussion}



\newpage
\section{Zusammenfassung}




\newpage
\section{Literaturverzeichnis}
\printbibliography







\end{document}