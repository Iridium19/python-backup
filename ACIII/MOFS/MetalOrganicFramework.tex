\documentclass[12pt, a4paper]{article}
\renewcommand*\contentsname{Inhaltsverzeichnis}
\usepackage[ngerman]{babel}
\usepackage{mathptmx}
\usepackage{blindtext}
\usepackage{emptypage}
\usepackage{wrapfig}
\usepackage[pdftex]{graphicx}
\usepackage{geometry}
\usepackage{setspace}
\usepackage{hyperref}
\usepackage[version=4,arrows=pgf-filled,
textfontname=sffamily,
mathfontname=mathsf]{mhchem}
\usepackage[table]{xcolor}
\usepackage{multirow}
\usepackage[table]{xcolor}
\usepackage{array}
\usepackage{mathcomp}
\usepackage{csquotes}
\usepackage[backend=biber,style=chem-acs,sorting=none]{biblatex}
\addbibresource{literatur.bib}  % Deine .bib-Datei einbinden

\DeclareCiteCommand{\cite}
  {\usebibmacro{prenote}}
  {\textsuperscript{\printfield{labelnumber}}}
  {\multicitedelim}
  {\usebibmacro{postnote}}


 \geometry{
 a4paper,
 total={170mm,257mm},
 left=25mm,
 top=25mm,
 }
\setstretch{1.213}


\newcommand{\datum}{\day.\month.\year}
\DeclareGraphicsExtensions{.pdf,.jpeg,.png,.jpg} 

\begin{document}


\begin{figure}
    \includegraphics[scale=0.14]{Universität_Bayreuth.svg.png}
\end{figure}


%Deckblatt

{\raggedright Universität Bayreuth\\  95447 Bayreuth}


\vspace{5cm}

\begin{center}
{\LARGE\bf{Anorganische Chemie III}} \\  
\vspace{1cm}
{\Large\bf{Überschrift}}\\
\vspace{0.5cm}
{\large Justus Friedrich\\}
{Studiengang: B.Sc. Chemie\\}
{4. Fachsemester}
\end{center}





\thispagestyle{empty}
\begin{center}
{\small Matrikelnummer: 1956010 \\
E–Mail:  bt725206@myubt.de}
\end{center}

\vspace{5cm}

\today


\newpage
%Inhaltsverzeichnis
\tableofcontents
\thispagestyle{empty}


%Teil 1
\newpage
\setcounter{page}{1}
\section{Einleitung}



\subsection{Einführung}
{MOFs gehören zu den Mikroporösen Materialien und werden nicht nur wegen ihrer hohen Gas adsorbier Fähigkeit und ihrer Möglichkeit als Katalysator 
immer Relevanter. Daher wird in der letzten Zeit sehr aktive Forschung an MOFs.\cite{ThomasHillman.2018}

}

\subsection{Ziel des Versuchs}
{Das Ziel dieses Versuchs ist es in einer 2er-Gruppe 4 verschiedene MOFs herzustellen. Diese werden dann miteinander verglichen und die Struktur mittels eines 
Pulverdiffraktogramms untersucht. Außerdem wird das Adsorbier-Verhalten in eine Iodkammer untersucht



}

\newpage
%Teil2
\renewcommand{\arraystretch}{1.3}
\section{Durchführung}
\subsection{Durchführung}
Es werden, um den entsprechenden MOF herzustellen, der Linker und das Metallchlorid werden in den Mengen die die Tabelle \ref{MOFmengen} beschreibt in 5 mL Dimethylformamid gelöst. 
Die Lösung wird dann in einen Autoklav überführt, und für 30 min bei 140 °C bei autogenen Druck zur Reaktion gebracht

\begin{table}[h!]
\caption{\textit{Zeigt die benötigten Reaktanten für die Synthese von Al-MIl-53-H, Fe-MIl-53-H, Al-MIl-53-NH$_2$, Fe-MIl-53-NH$_2$}.\cite{Skript}}
\begin{center}
\begin{tabular}{|p{4cm}|p{4cm}|p{6cm}|}
    \hline
    MOF & Einwaage $MCL_3$ & Einwaage Linker \\
    \hline
    Al-MIl-53-H & 123.4 mg $AlCl_3$ & 172.3 mg Terephtalsäure \\
    \hline
    Fe-MIl-53-H & 138.2 mg $FeCl_3$ & 172.3 mg Terephtalsäure\\
    \hline
    Al-MIl-53-NH$_2$ & 123.4 mg $AlCl_3$ & 187.8 mg Aminoterephtalsäure\\
    \hline
    Fe-MIl-53-NH$_2$ & 138.2 mg $FeCl_3$ & 187.8 mg Aminoterephtalsäure\\
    \hline

\end{tabular}

\end{center}
{
Das entstandene Produkt wird in ein Zentrifugenglas überführt, und für 5 min bei 5000 
Umdrehungen die Minute zentrifugiert. Das DMF wird dekantiert und entsorgt. 
Anschließend wird das Produkt mit Wasser 
aufgeschwemmt und erneut zentrifugiert und dekantiert. 
Der Prozess wird mit Ethanol wiederholt. Nachdem das Ethanol abdekantiert ist, wird das Produkt im Trockenschrank getrocknet.\cite{Skript}
\vspace{0.2cm}}\\
{
Vom dem MOFs werden XRDs augenommen. Anschließend wird eine Spatelspitze der MOFs in ein Rollglas gegeben und mit 1-2 Iodkügelchen auf 50 °C 
erhitzt. Dabei wird das Verhalten beobachtet.
}






\label{MOFmengen}

\end{table}


\newpage
\section{Auswertung}
\subsection{Ergebnisse}
\subsection{Diskussion}



\newpage
\section{Zusammenfassung}
\cite{Skript}



\newpage
\section{Literaturverzeichnis}
\printbibliography


\end{document}