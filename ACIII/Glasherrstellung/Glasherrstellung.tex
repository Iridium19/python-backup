\documentclass[12pt, a4paper]{article}
\renewcommand*\contentsname{Inhaltsverzeichnis}
\usepackage[ngerman]{babel}
\usepackage{mathptmx}
\usepackage{blindtext}
\usepackage{emptypage}
\usepackage{wrapfig}
\usepackage[pdftex]{graphicx}
\usepackage{geometry}
\usepackage{setspace}
\usepackage{hyperref}
\usepackage[version=4,arrows=pgf-filled,
textfontname=sffamily,
mathfontname=mathsf]{mhchem}
\usepackage[table]{xcolor}
\usepackage{multirow}
\usepackage[table]{xcolor}
\usepackage{array}
\usepackage{float}
\usepackage{mathcomp}
\usepackage{csquotes}
\usepackage[backend=biber,style=chem-acs,sorting=none]{biblatex}
\addbibresource{literatur.bib}  % Deine .bib-Datei einbinden

\DeclareCiteCommand{\cite}
  {\usebibmacro{prenote}}
  {\textsuperscript{\printfield{labelnumber}}}
  {\multicitedelim}
  {\usebibmacro{postnote}}


 \geometry{
 a4paper,
 total={170mm,257mm},
 left=25mm,
 top=25mm,
 }
\setstretch{1.213}


\newcommand{\datum}{\day.\month.\year}
\DeclareGraphicsExtensions{.pdf,.jpeg,.png,.jpg} 

\begin{document}


\begin{figure}
    \includegraphics[scale=0.14]{Universität_Bayreuth.svg.png}
\end{figure}


%Deckblatt

{\raggedright Universität Bayreuth\\  95447 Bayreuth}


\vspace{5cm}

\begin{center}
{\LARGE\bf{Anorganische Chemie III}} \\  
\vspace{1cm}
{\Large\bf{Glassherstellung}}\\
\vspace{0.5cm}
{\large Justus Friedrich\\}
{Studiengang: B.Sc. Chemie\\}
{4. Fachsemester}
\end{center}





\thispagestyle{empty}
\begin{center}
{\small Matrikelnummer: 1956010 \\
E–Mail:  bt725206@myubt.de}
\end{center}

\vspace{5cm}

\today


\newpage
%Inhaltsverzeichnis
\tableofcontents
\thispagestyle{empty}


%Teil 1
\newpage
\setcounter{page}{1}
\section{Einleitung}



\subsection{Motivation}
Gläser besitzen in der Regel interessante physikalische und chemische Eigenschaften. Diese resultieren aus ihrer amorph strukturierten Anordnung, 
bei der keine langfristige, regelmäßige Kristallstruktur vorliegt. In diesem Experiment sollen Gläser mit unterschiedlichen Konzentrationen von Netzwerkbildnern und Netzwerkwandlern 
hergestellt werden. Anschließend wird der Verknüpfungsgrad der Netzwerkbildner analysiert, um Rückschlüsse auf die Struktur und Eigenschaften des Glases ziehen zu können. \cite{Skript}

\newpage
%Teil2
\section{Durchführung}
\subsection{Synthese der Verschieden Gläser}
Es werden sieben verschiedene Glaszusammensetzungen hergestellt. Die dafür benötigten Massen der Ausgangsstoffe, um 2 g Glass zu bekommen, werden der Tabelle \ref{Verhältnisse} entnommen. Die jeweiligen Komponenten werden sorgfältig miteinander vermörsert und anschließend in Quarztiegel überführt.
\noindent
Die Proben werden zunächst über 2 Stunden auf 200 °C erhitzt und bei dieser Temperatur für weitere 2 Stunden gehalten. Danach erfolgt eine weitere Aufheizung auf 800 °C über 2 Stunden, gefolgt von einem Halten bei dieser Temperatur für weitere 2 Stunden.
\noindent
Anschließend werden die Gläser durch Abschrecken bei Raumtemperatur (Quenching) verfestigt. Dazu werden sie in einen Exsikkator unter Schutzglas überführt.


\begin{table}[!h]
  \caption{Zeigt die Mol Verhältnisse der Produkte im Glas, und die dafür nötigen Eduktmassen und deren Mol Anzahl. Die Berrechungen für die Mol-Anzahl sind in Gleichung (1) und (2) dargestellt.}
  \begin{center}
    \begin{tabular}{|>{\centering\arraybackslash}p{2.3cm}|>{\centering\arraybackslash}p{1cm}|>{\centering\arraybackslash}p{1cm}|>{\centering\arraybackslash}p{1cm}|>{\centering\arraybackslash}p{1cm}|>{\centering\arraybackslash}p{1cm}|>{\centering\arraybackslash}p{1cm}|>{\centering\arraybackslash}p{1cm}|>{\centering\arraybackslash}p{1cm}|}
      \hline
      \rowcolor{gray}
      \cellcolor{lightgray}Mol\% \ce{Na2O} & 30\% & 35\% & 40\% & 45\% & 50\% & 55\% & 60\% & 70\% \\
      \hline
      \rowcolor{yellow}
       \cellcolor{lightgray}Masse \ce{Na2CO3} [g]&0.54&0.66&0.77&0.90&1.04&1.19&1.35&1.73 \\
      \hline
       \cellcolor{lightgray}Mol \ce{Na2CO3} [mmol]&5.09&6.23&7.26&8.49&9.81&11.2&12.7&16.3 \\
      \hline
      \rowcolor{gray}
       \cellcolor{lightgray}Mol\% \ce{P2O5} & 70\% & 65\% & 60\% & 55\% & 50\% & 45\% & 40\% & 30\% \\
      \hline
      \rowcolor{yellow}      
       \cellcolor{lightgray}Masse \ce{NH4H2PO4} [g]&2.73&2.61&2.51&2.39&2.26&2.11&1.95&2.61 \\
      \hline
       \cellcolor{lightgray}Mol \ce{NH4H2PO4} [mmol]&23.7&22.7&21.8&20.8&19.6&18.34&16.9&13.9 \\
      \hline
    \end{tabular}
  \end{center}

  \label{Verhältnisse}
\end{table}

\subsection{Gleichungen zur Berechnung}
\begin{equation}
  \frac{2 g}{M(\ce{Na2O})+\frac{mol\%  (\ce{P2O5})}{mol\%  (\ce{Na2O})}\cdot M(\ce{P2O5})}= n(\ce{Na2CO3})
\end{equation}

\begin{equation}
  n(\ce{NH4H2PO4})= 2 \cdot n(\ce{Na2CO3})
\end{equation}



\newpage
\section{Auswertung}
\subsection{Ergebnisse}
\subsection{Diskussion}



\newpage
\section{Zusammenfassung}




\newpage
\section{Literaturverzeichnis}
\printbibliography







\end{document}