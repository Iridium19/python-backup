\documentclass[12pt, a4paper]{article}
\renewcommand*\contentsname{Inhaltsverzeichnis}
\usepackage[ngerman]{babel}
\usepackage{mathptmx}
\usepackage{blindtext}
\usepackage{emptypage}
\usepackage{wrapfig}
\usepackage[pdftex]{graphicx}
\usepackage{geometry}
\usepackage{setspace}
\usepackage[version=4,arrows=pgf-filled,
textfontname=sffamily,
mathfontname=mathsf]{mhchem}
\usepackage{hyperref}
\usepackage{mathcomp}
\usepackage{multirow}
\usepackage{array}
\usepackage{mathcomp}
\usepackage{csquotes}
\usepackage[backend=biber,style=chem-acs,sorting=none]{biblatex}
\addbibresource{literatur.bib}  % Deine .bib-Datei einbinden

\DeclareCiteCommand{\cite}
  {\usebibmacro{prenote}}
  {\textsuperscript{\printfield{labelnumber}}}
  {\multicitedelim}
  {\usebibmacro{postnote}}


 \geometry{
 a4paper,
 total={170mm,257mm},
 left=25mm,
 top=25mm,
 }
\setstretch{1.213}


\newcommand{\datum}{\day.\month.\year}
\DeclareGraphicsExtensions{.pdf,.jpeg,.png,.jpg} 

\begin{document}


\begin{figure}
    \includegraphics[scale=0.14]{Universität_Bayreuth.svg.png}
\end{figure}


%Deckblatt

{\raggedright Universität Bayreuth\\  95447 Bayreuth}


\vspace{5cm}

\begin{center}
{\LARGE\bf{Anorganische Chemie III}} \\  
\vspace{1cm}
{\Large\bf{Gelkristallisation und Chemischer Transport}}\\
\vspace{0.5cm}
{\large Justus Friedrich\\}
{Studiengang: B.Sc. Chemie\\}
{4. Fachsemester}
\end{center}





\thispagestyle{empty}
\begin{center}
{\small Matrikelnummer: 1956010 \\
E-Mail:  bt725206@myubt.de}
\end{center}

\vspace{5cm}
\begin{center}
  \today
\end{center}



\newpage
%Inhaltsverzeichnis
\tableofcontents
\thispagestyle{empty}


%Teil 1
\newpage
\setcounter{page}{1}
\section{Einleitung}



\subsection{Einführung}
{Einkristalle sind für die Charakterisierung sehr relevant, da damit sich die Strukturen nachvollziehen lassen. Allerdings ist das recht 
schwer gute Einkristalle von Schwerlöslichen Salzen zu erhalten. Wenn die einzelnen Ausgangsstoffe der Verbindung zur verfügen stehen, kann mittels 
der Gelkristallisation Einkristalle gezüchtet werden. Wenn allerdings nur eine unsaubere Probe des kristallinen Stoffes vorliegt, kann, soweit ein Transportmittel vorhanden, 
der Stoff chemisch transportiert werden. Dabei entsteht ein sauberer Einkristall, während die Verunreinigungen zurückbleiben.
}

\subsection{Ziel des Versuchs}
{Es soll in einem Agar-Agar-Gel und in einem Kieselgel Kristalle gezüchtet werden. Außerdem soll ein Stoff chemisch transportiert werden. 
Anschließend werden die Kristalle unter einem Mikroskop untersucht. Dabei sollen die Eigenschaften der Kristalle bestimmt werden, und verglichen werden. 

}






\newpage
%Teil2
\section{Durchführung}
\subsection{Herstellung eines Agar-Agar-Gels und Kristallwachstum in diesem}
{Es werden 24.5 mL VE-Wasser mit 0.5 g Agar-Agar vermischt, und solange gekocht, dass sich das Agar-Agar vollständig gelöst hat. Anschließend wird 25 mL einer warme 1 molare $CaCl_2$-Lösung hinzugegeben. 
Diese wurde aus 3.676 g (25.0 mmol) $CaCl_2$ und 25 mL Wasser hergestellt. Die entstehende Mischung wird daraufhin auf vier Reagenzgläser aufgeteilt und erstarren lassen. Nachdem das Gel vollständig erstarrt ist, 
wird eine 0.5 molare Lösung von $Na_2SO_3$ (1.90 g/0.015 mol in 30 mL Wasser) darauf gegeben. Nach Zwei Wochen werden die entstehende Kristalle aus dem Gel genommen und unter einem Mikroskop untersucht.\cite{Skript}

}


\subsection{Herstellung eines Kieselgels und Kristallwachstum in diesem}
{Es werden 0.684 g (4.56 mmol) Weinsäure in 20 mL 1 molarer Essigsäure gelöst. Anschließt wird langsam Natriummetasilikatlösung hinzugetropft, bis sich ein pH-Wert von 6 eingestellt hat. Dies benötigt ca. 8 mL. Wenn sich der pH-Wert 
eingestellt hat, dann wird die Lösung auf 3 Reagenzgläser aufgeteilt und gewartet bis das Gel vollständig erstarrt ist. Anschließend wird mit einer 0.22 molare $CuSO_4$-Lösung (1.66 g/6.66 mmol $CuS0_4\cdot 5 H_2O$ in 30 mL Wasser) 
überschichtet. Nach einer bis zwei Woche werden die entstandene Kristalle entnommen und unterm Mikroskop untersucht.\cite{Skript}
}


\subsection{Chemischer Transport}
\subsubsection{\texorpdfstring{Chemischer Transport mit $Ag_2S$}{Chemischer Transport mit Ag2S}}
{In eine ausgeheizte Quarzampulle werden 70 mg (0.28 mmol) $Ag_2S$ und 2 Iod-Kügelchen gegeben. Daraufhin wird in der Ampulle ein Vakuum gezogen und abgeschmolzen. Daraufhin wird die Ampulle in einen Ofen gelegt, der ein Temperaturgradient von 
825 °C$\rightarrow$700 °C besitzt. Nach zwei Wochen werden die Kristalle unter dem Mikroskop betrachtet.\cite{Skript}



}
\subsubsection{\texorpdfstring{Chemischer Transport mit $SnS_2$}{Chemischer Transport mit SnS2}}
{In eine ausgeheizte Quarzampulle werden 70 mg (0.38 mmol) $SnS_2$ und 2 Iod-Kügelchen gegeben. Daraufhin wird in der Ampulle ein Vakuum gezogen und abgeschmolzen. Daraufhin wird die Ampulle in einen Ofen gelegt, der ein Temperaturgradient von 
700 °C$\rightarrow$550 °C besitzt. Nach zwei Wochen werden die Kristalle unter dem Mikroskop betrachtet.\cite{Skript}
}




\newpage
\section{Auswertung}
\subsection{Ergebnisse}
\subsection{Diskussion}



\newpage
\section{Zusammenfassung}
\cite{Skript}



\newpage
\section{Literaturverzeichnis}
\printbibliography







\end{document}